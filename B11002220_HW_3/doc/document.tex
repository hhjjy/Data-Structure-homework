\documentclass{article}
\usepackage{graphicx}
\usepackage{xcolor}
\usepackage{listings}
\usepackage[ruled,vlined]{algorithm2e}
\usepackage{tikz}
\usepackage{ulem}

\usepackage{xeCJK}
    \xeCJKsetup{AutoFakeBold=true, AutoFakeSlant=true}
    \setCJKmainfont{標楷體}
    \setmainfont{Times New Roman}

\definecolor{backgroundColour}{rgb}{0.95,0.95,0.92}
\definecolor{mGray}{rgb}{0.5,0.5,0.5}
\definecolor{mGreen}{rgb}{0.1,0.5,0.1}

\definecolor{mPurple}{rgb}{0.58,0,0.82}
\definecolor{backgroundColour}{rgb}{0.95,0.95,0.92}

%
% Various Helper Commands
%

% Useful for algorithms
\newcommand{\alg}[1]{\textsc{\bfseries \footnotesize #1}}

% For derivatives
\newcommand{\deriv}[1]{\frac{\mathrm{d}}{\mathrm{d}x} (#1)}

% For partial derivatives
\newcommand{\pderiv}[2]{\frac{\partial}{\partial #1} (#2)}

% Integral dx
\newcommand{\dx}{\mathrm{d}x}

% Alias for the Solution section header
\newcommand{\solution}{\textbf{\large Solution}}

% Probability commands: Expectation, Variance, Covariance, Bias
\newcommand{\E}{\mathrm{E}}
\newcommand{\Var}{\mathrm{Var}}
\newcommand{\Cov}{\mathrm{Cov}}
\newcommand{\Bias}{\mathrm{Bias}}

\lstdefinestyle{CStyle}{
    backgroundcolor=\color{backgroundColour},   
    commentstyle=\color{mGreen},
    keywordstyle=\color{magenta},
    numberstyle=\tiny\color{mGray},
    stringstyle=\color{mPurple},
    basicstyle=\footnotesize,
    breakatwhitespace=false,         
    breaklines=true,                 
    captionpos=b,                    
    keepspaces=true,                 
    numbers=left,                    
    numbersep=5pt,                  
    showspaces=false,                
    showstringspaces=false,
    showtabs=false,                  
    tabsize=2,
    language=C
}

%%%% USERFUL FUNCTION


\begin{document}

	\begin{titlepage}
		\centering
		\vspace*{1cm}
		\Huge
		\textbf{Data Structure}
		
		\vspace{2cm}
		\LARGE
		\textbf{Homework3:}Binary search tree implementation
		
		\vspace{2cm}
		\textbf{Author:} B11002220 CHI-CHUN, LO
		
		\vspace{1cm}



		\textbf{Date: }\today
		
		\vfill
		\vspace{2cm}

    	\includegraphics[width=1\textwidth]{logo.png}
		
		
	\end{titlepage}
	\pagebreak
	% 目錄
	\tableofcontents
	
	
	% Problem page 
	\pagebreak
	\section{Problem}	
	Write a program to process binary search trees. including the following operations
		\begin{enumerate}
			\item Insert a sequence of characters into a binary search tree. one character at a time.
			\item Delete all leaf nodes of the tree built in the previous operation. 
		\end{enumerate}
	After each operation. the program prints out all keys in increasing order for the inorder traversal. 
	\subsection{Definition}
	簡單來說,這個題目要求我們透過二元搜尋樹完成插入(Insert)和刪除葉子節點(DeleteLeafs)這兩個操作。我們將逐步說明這兩個操作的含義和要求。
	首先,我們先定義一個二元搜尋樹的結構。二元搜尋樹的每個節點包含儲存自身資料的值,以及左右兩個子節點的指標。
	在每個操作之後,程序將按照中序遍歷的方式,打印出所有鍵值的递增順序。
		\begin{lstlisting}[style=Cstyle]
			typedef struct node* treePointer;
			typedef struct node
			{
				int data;
				treePointer left;
				treePointer right;
			} Node;
		\end{lstlisting}
		值得注意的是,二元搜尋樹有以下限制與要求:
		\begin{itemize}
			\item 所有左子樹的節點值必須小於父節點的值。
			\item 所有右子樹的節點值必須大於父節點的值。
			\item 每個節點的值必須是唯一的,不允許重複值。
		\end{itemize}

		\vspace{0.6cm}
      	針對第一個操作,對二元樹(BST)插入節點。我們可以定義一個函數InsertNode,它的輸入是這個結構的root,輸出是root節點,透過遞迴將資料插入到正確位置:
	     \begin{lstlisting}[style=Cstyle]
			treePointer insertNode(treePointer root, char data)
		\end{lstlisting}
		\vspace{0.6cm}

		針對第二個操作,將二元樹(BST)的葉子(leaf)節點都刪除。我們可以定義一個函數DeleteLeafs,它的輸入是這個結構的root,輸出是root節點,在遞迴的過程中若是發現葉子節點就進行刪除的動作:
		\begin{lstlisting}[style=Cstyle]
			treePointer deleteLeaf(treePointer root)
		\end{lstlisting}
		 

		根據作業規定,我們將使用一個名為Input.txt的檔案作為輸入文件,Output.txt作為輸出文件檔案。
		每當Input.txt的輸入一行文字時,就會輸出其inorder排序的結果與刪除子葉節點的結果。
		我們可以針對整個程式的main.c大致的將整個程式的流程設計好,如下:
		\begin{lstlisting}[style=Cstyle]
			int main()
			{
				clear_file();
				// Set input file 
				FILE* inputFile;
				if (fopen_s(&inputFile, "Input.txt", "r") != 0)
				{
					fprintf(stderr, "Failed to open Input.txt\n");
					return 1;
				}
			
				// Read each line
				char line[100];
				while (fgets(line, sizeof(line), inputFile) != NULL)
				{
					line[strcspn(line, "\n")] = '\0'; // delete  \n 
					treePointer BST = NULL;
					// Read a char and send it into BST 
					// first opeation 
					for (int i = 0; line[i]; i++)
					{
						BST = insertNode(BST, line[i]);
					}
					inorderTraversal(BST); 
					printf("\n");
					// second opeartion 
					deleteLeaf(BST); 
					inorderTraversal(BST);
					printf("\n");
				}
				fclose(inputFile);
				return 0;
			}
		\end{lstlisting}

		\subsection{InsertNode}
		對於\alg{insertNode()}函數,我們會根據數據大小往右或往左遞迴找到合適擺放新資料的位置。另外對於
		\alg{insertNode()}函數,我們會考慮到當插入的資料已經存在於BST中,則會直接跳過。

		\begin{algorithm}[H]
			\SetAlgoLined
			\SetKwInOut{Input}{Input}
			\SetKwFunction{createNode}{createNode}
			\SetKwFunction{insertNode}{insertNode}
			
			\SetKwProg{Fn}{Function}{:}{}
			\Fn{\insertNode{root, data}}{
				\Input{root: treePointer, data: char}
				\If{root is NULL}{
					\KwRet \createNode{data}\;
				}
				\If{root->data is less than data}{
					root->right = \insertNode{root->right, data}\;
				}
				\If{root->data is greater than data}{
					root->left = \insertNode{root->left, data}\;
				}
				\KwRet root\;
			}
			
			\caption{Insert a node into a binary search tree}
			\end{algorithm}
		\subsection{DeleteLeaf}
		這段程式碼的目的是刪除二元搜尋樹中的葉子節點
		,透過遞迴地遍歷樹的每個節點,並在遇到葉子節點時將其釋放。這樣可以逐步刪除所有的葉子節點,最終達到刪除目標的效果。
		\begin{algorithm}[h]
			\SetAlgoLined
			\SetKwInOut{Input}{Input}
			\SetKwFunction{deleteLeaf}{deleteLeaf}
			
			\Input{root: treePointer}
			\KwResult{root after deleting leaf nodes}
			
			\SetKwProg{Fn}{Function}{:}{}
			\Fn{\deleteLeaf{root}}{
				\If{root is not null}{
					\If{both root->left and root->right are null}{
						release current node\;
						\KwRet null\;
					}
					\deleteLeaf{root->left}\;
					\deleteLeaf{root->right}\;
				}
				\KwRet root\;
			}
			
			\caption{Delete leaf nodes in a binary search tree}
			\end{algorithm}
		\subsection{inorderTraversal}
		這段程式碼的目的是對一個二元搜尋樹進行inorder traversal。
		\begin{algorithm}[h]
			\SetKwInOut{Input}{Input}

			\Input{root: treePointer}
			\KwResult{root you gave before}

			\SetAlgoLined
			\SetKwFunction{inorderTraversal}{inorderTraversal}
			
			\SetKwProg{Fn}{Function}{:}{}
			\Fn{\inorderTraversal{root}}{
				\If{root is not null}{
					\inorderTraversal{root->left}\;
					print root->data\;
					\inorderTraversal{root->right}\;
				}
			}
			
			\caption{Inorder traversal of a binary search tree}
			\end{algorithm}

	\pagebreak
	% Code
	\section{Code}	
		下面是main.c完整的程式碼,我有額外編寫一個常用的
		\texttt{useful.h}檔案,裡面有許多常用的函式,裡面有重新定義printf的程式
		,針對不同環境下的C編譯器設定。
		\subsection{useful.h}
			\begin{lstlisting}[style=CStyle]
				#ifndef __useful__
				#define __useful__

				#include "stdio.h"
				#include "stdlib.h"
				#include "time.h"

				/**
				* @brief Define the filename for the output file.
				*/
				#define OUTPUT_FILE "Output.txt"

				#ifdef __PRINTF_TO_FILE__BY_VISUAL_STUDIO
				/**
				* @brief Clear the content of the output file.
				*/
				#define clear_file() \
						do { \
							FILE *file; \
							if(fopen_s(&file, OUTPUT_FILE, "w") == 0) { \
								if (file != NULL) { \
									fprintf(file,""); \
									fclose(file); \
								} \
							} \
						} while (0)

				/**
				* @brief Print content to the output file.
				*/
				#define printf(...) \
						do { \
							FILE *file; \
							if(fopen_s(&file, OUTPUT_FILE, "a") == 0) { \
								fprintf(file,__VA_ARGS__); \
								fclose(file); \
							} \
						} while (0)

				#endif

				/**
				* @brief Standard C fopen
				*/
				#ifdef __PRINTF_TO_FILE__
				/**
					* @brief Clear the content of the output file.
					*/
				#define clear_file() \
						do { \
							FILE *file = fopen(OUTPUT_FILE, "w"); \
							if (file != NULL) { \
								fclose(file); \
							} \
						} while (0)

					/**
					* @brief Print content to the output file.
					*/
				#define printf(...) \
						do { \
							FILE *file = fopen(OUTPUT_FILE, "a"); \
							if (file != NULL) { \
								fprintf(file, __VA_ARGS__); \
								fclose(file); \
							} \
						} while (0)

				#endif

				#ifndef  __PRINTF_TO_FILE__BY_VISUAL_STUDIO && __PRINTF_TO_FILE__
					/**
					* @brief If __PRINTF_TO_FILE__ is not defined, do nothing.
					*/
				#define clear_file() \
						do { \
						} while (0)
				#endif

				// Display the entire array
				#define PRINT_ARR(ARR, SIZE) \
					do \
					{ \
						for (int i = 0; i < SIZE; i++) \
							printf("%d ", ARR[i]); \
						printf("\n"); \
					} while (0)

				// Swap
				#define SWAP(a, b, t) (t = a, a = b, b = t)

				/**
				* @brief Compare two numbers. Returns 1 if a > b, 0 if a = b, -1 if a < b.
				* @param a
				* @param b
				*/
				#define COMPARE(a, b) (a > b ? 1 : (a < b ? -1 : 0))

				// Allocate memory
				#define MALLOC(arr, size) \
					do \
					{ \
						if (!(arr = malloc(size))) \
						{ \
							fprintf(stderr, "Malloc error\n"); \
							exit(EXIT_FAILURE); \
						} \
					} while (0)

				#define REMALLOC(arr, size) \
					do \
					{ \
						if (!(arr = realloc(arr, size))) \
						{ \
							fprintf(stderr, "Realloc error\n"); \
							exit(EXIT_FAILURE); \
						} \
					} while (0)

				// Display the execution time of a function (input: void)
				#define TIME_FUNCTION(func)                                                         \
					do                                                                              \
					{                                                                               \
						clock_t start = clock();                                                    \
						func();                                                                     \
						clock_t end = clock();                                                      \
						double cpu_time_used = ((double)(end - start)) / CLOCKS_PER_SEC;            \
						printf("Function %s takes %f seconds to execute.\n", #func, cpu_time_used); \
					} while (0)

				#endif

			\end{lstlisting}
		\subsection{main.c}
		一開始的定義是為了調用useful.h重新定義好的printf,可以讓printf到檔案中,而不是在螢幕上顯示。
		\begin{lstlisting}[style=CStyle]
		/**
		* @file BST.c
		* @author Leo (you@domain.com)
		* @brief HW3 BST
		* @version 0.1
		* @date 2023-11-17
		*
		* @copyright Copyright (c) 2023
		*
		*/
		#define __PRINTF_TO_FILE__BY_VISUAL_STUDIO

		#include "useful.h"
		typedef struct node *treePointer;
		typedef struct node
		{
			int data;
			treePointer left;
			treePointer right;
		} Node;
		treePointer createNode(char data)
		{
			treePointer newNode;
			MALLOC(newNode, sizeof(*newNode));
			newNode->data = data;
			newNode->left = NULL;
			newNode->right = NULL;
			return newNode;
		}
		/**
		* @brief Delete leaf nodes in a binary tree.
		* @param root The root node of the binary tree.
		* @return treePointer The modified root node of the binary tree.  
		*/
		treePointer deleteLeaf(treePointer root)
		{
			if (root)
			{
				if (root->left == NULL && root->right == NULL) // Check if it is a leaf node
				{
					free(root);  // Free the memory space of the leaf node
					return NULL; // Return a null pointer to indicate the deletion of the leaf node
				}
				// Recursively delete leaf nodes in the left subtree and right subtree
				root->left = deleteLeaf(root->left);
				root->right = deleteLeaf(root->right);
			}
			return root; // Return the modified root node of the binary tree
		}

		/**
		* @brief Insert a node at the specified position.
		*
		* @param root The current node.
		* @param data The character data to be inserted.
		* @return treePointer
		* @note: There are two possible situations that can occur.
		*          1. Return the address of the new node if the root node is empty!
		*          2. Return root to pass the previously  address to root->right or root->left.
		*/
		treePointer insertNode(treePointer root, char data)
		{
			// Handle the case where the current node is empty
			if (root == NULL)
			{
				return createNode(data);
			}
			// Search to the right
			if (root->data < data)
			{
				root->right = insertNode(root->right, data);
			}
			// Search to the left
			if (root->data > data)
			{
				root->left = insertNode(root->left, data);
			}
			// Make sure that right & left store the current root node to connect successfully
			return root;
		}

		/**
		* @brief Search for a node containing the specified data in a binary search tree.
		* @param root The root node of the binary search tree.
		* @param data The data to search for.
		* @return treePointer a pointer to the node containing the search data, or NULL if not found.
		*/
		treePointer search(treePointer root, char data)
		{
			if (!root || root->data == data) // if current node or data find then return current node   
				return root;
			if (root->data < data) // search from right 
				return search(root->right, data);
			else // search from left 
				return search(root->left, data);
		}
		/**
		* @brief Perform an inorder traversal of a binary tree.
		*
		* @param root The root node of the binary tree.
		* @return void
		*/
		void inorderTraversal(treePointer root)
		{
			if (root != NULL)
			{
				inorderTraversal(root->left);
				printf("%c", root->data);
				inorderTraversal(root->right);
			}
		}

		void printTree(treePointer root, int level)
		{
			if (root == NULL)
			{
				return;
			}

			printTree(root->right, level + 1);

			for (int i = 0; i < level; i++)
			{
				printf("    ");
			}
			printf("%c\n", root->data);

			printTree(root->left, level + 1);
		}
		int main()
		{
			clear_file();
			// Set input file 
			FILE* inputFile;
			if (fopen_s(&inputFile, "Input.txt", "r") != 0)
			{
				fprintf(stderr, "Failed to open Input.txt\n");
				return 1;
			}

			// Read each line
			char line[100];
			while (fgets(line, sizeof(line), inputFile) != NULL)
			{
				line[strcspn(line, "\n")] = '\0'; // delete  \n 
				treePointer BST = NULL;
				// Read a char and send it into BST 
				// first opeation 
				for (int i = 0; line[i]; i++)
				{
					BST = insertNode(BST, line[i]);
				}
				inorderTraversal(BST); 
				printf("\n");
				// second opeartion 
				deleteLeaf(BST); 
				inorderTraversal(BST);
				printf("\n");
			}
			fclose(inputFile);
			return 0;
		}
		\end{lstlisting}
		
	\section{Result}
	在這邊我們會針對不同的例子去測試程式的穩定性、正確性。
	\begin{itemize}
		\item \textbf{example1:} 討論輸入順序數字
		
		\textbf{Input:}
		\begin{lstlisting}[style=Cstyle]
			123
		\end{lstlisting}
		\textbf{Output:}
		\begin{lstlisting}[style=Cstyle]
			123
			12				
		\end{lstlisting}
		解釋:第一列是插入後以Inorder的順序輸出結果,第二列是刪除葉子節點的輸出結果。
		\item \textbf{example2:} 討論輸入的節點有兩個葉子節點 \\
		\textbf{Input:}
		\begin{lstlisting}[style=Cstyle]
			213		
		\end{lstlisting}
		\textbf{Output:}
		\begin{lstlisting}[style=Cstyle]
			123
			2							
		\end{lstlisting}
		解釋:第一列是插入後以Inorder的順序輸出結果,第二列是刪除葉子節點的輸出結果。確實葉子節點1與3都被刪除了。

		
		\item \textbf{example3:} 重複數字輸入造成是否會輸入到二元搜索樹內?
		
		\textbf{Input:}
		\begin{lstlisting}[style=Cstyle]
			223311
		\end{lstlisting}
		\textbf{Output:}
		\begin{lstlisting}[style=Cstyle]
			123
			2		
		\end{lstlisting}
		第一列是插入後以Inorder的順序輸出結果,第二列是刪除葉子節點的輸出結果。可以看出重複的數字會被拋棄掉!

		\item \textbf{example4:} 測試只有輸入一個數字時,刪除葉子節點是否會有問題發生?\\
		\\
		如果依照之前設計的程式葉子節點會被刪除,但實際上我們不能把root 節點刪除,程式需要把這個狀態排除,所以下面是我針對root節點被刪除的問題做的修正。
		\\ \\
		\textbf{Imporve program:}
		\begin{lstlisting}[style=Cstyle]
		/**
		* @brief Delete leaf nodes in a binary tree.
		* @param root The root node of the binary tree.
		* @param level it save the current root's level . 
		* @return treePointer The modified root node of the binary tree.  
		*/
		treePointer deleteLeaf(treePointer root,int level)
		{
			if (root)
			{
				if (root->left == NULL && root->right == NULL && level!= 0) // Check if it is a leaf node
				{
					free(root);  // Free the memory space of the leaf node
					return NULL; // Return a null pointer to indicate the deletion of the leaf node
				}
				// Recursively delete leaf nodes in the left subtree and right subtree
				root->left = deleteLeaf(root->left,level+1);
				root->right = deleteLeaf(root->right, level + 1);
			}
			return root; // Return the modified root node of the binary tree
		}
		
		\end{lstlisting}
		\textbf{Input:}
		\begin{lstlisting}[style=Cstyle]
			2
		\end{lstlisting}
		
		\textbf{Output:}
		\begin{lstlisting}[style=Cstyle]
			2
			2
		\end{lstlisting}
		第一列是插入後以Inorder的順序輸出結果,第二列是刪除葉子節點的輸出結果。
		這個例子本身是root節點同時也是葉子節點,可以有兩種結果發生,一是把root節點刪除,二是保留root節點。這邊理應把root節點保留避免發生錯誤。這邊是做成不刪除。

		

	\end{itemize}
	
	
	% Discussion and Conclusion
	\section{Discussion}

	在本節中,我們對二元搜尋樹(Binary Search Tree,BST)的整體輸入和輸出進行了簡要說明。只要當前的葉子節點不是BST樹的根節點,就可以進行刪除葉子節點的操作。在插入過程中,重複的節點會被拋棄,確保BST中不會存儲重複的數字。

	接下來,我們討論了程式的效率,其中 n 表示當前節點的數量。

	\begin{itemize}
		\item 插入操作的時間複雜度為 $O(\log n)$
		\item 刪除操作的時間複雜度為 $O(n)$。
	\end{itemize}

	為了進一步提高效率,我們可以檢測並跳過重複數字的插入。在原始設計中,相同的數字不會被插入,但仍會執行對數時間的運算。我們可以修改程式碼,直接跳過插入操作。

	\begin{lstlisting}[style=Cstyle]
	// 新增accessed_before偵測是否有使用過這個文字!
	char accessed_before[256] = {0};
	for (int i = 0; line[i]; i++) {
	if (accessed_before[line[i]] == 0) {
	accessed_before[line[i]]++ ;
	BST = insertNode(BST, line[i]);
	} else {
	continue ;
	}
	}
	\end{lstlisting}

	\section{Conclusion}
	這次的作業,我花的比較多的時間在重定義的設計上,以及程式碼的優化上。因為之前的作業中常常會用到輸入、輸出的功能,格式
	和內容都比較固定,所以對於這次的作業,我花了很多時間在研究如何設計出符合作業需求的輸入、輸出格式。
	\\\\
	在程式碼的優化方面,我發現了程式碼中重複的節點檢查,以及重複的數字檢查。這些檢查可以被合併,以提高程式效率。這些都是非常有用的技能,在日後的程式開發中會派上用場。
	
\end{document}
